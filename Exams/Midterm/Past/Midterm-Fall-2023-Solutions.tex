\documentclass{./../../../Latex/tests}
\begin{document}
\thispagestyle{plain}
\myheader{Fall 2023 Midterm Exam: Solutions}
\rhead{Fall 2023 Midterm Exam}
\vspace{0.5em}
\testHeader{110}{30}

\begin{enumerate}	

%%%%%%%%%%%%%% Question 1
\item (10 pts) Answer the following questions
\begin{enumerate}
% Part 
\item (1 pt) Consider two sets $A$ and $B$, where $A$ is the set of all odd real numbers and $B$ is the set of all even real numbers. What is the union of $A$ and $B$? 
$$ A \cup B = \text{Set of all integers} $$
% Part
\item (1 pt) Expand the following summation expression: $$\sum_{j=1}^2  \sum_{i=1}^2 X_i Y_j = \sum_{j=1}^2  \left(\sum_{i=1}^2 X_i Y_j \right) = \sum_{j=1}^2 \left( X_1 Y_j + X_2 Y_j \right) = X_1 Y_1 + X_2 Y_1 + X_1 Y_2 + X_2 Y_2 $$
 \vspace{1em}
 % Part
\item (1pt) Does the inverse of the function $f: \mathbb{R} \rightarrow \mathbb{R}$, $f(x)=x^2$ exist? If yes, find the inverse. 

No, since this function is non-monotonic, the inverse does not exist. \\

% Part
\item (1.5 pts) Is the following function continuous? Is it differentiable? Justify your answer.
$$ f(x) = 
    \begin{cases} 
    x^2 & \text{if } x < 0 \\
    x & \text{if } x \geq 0 
    \end{cases} $$

   The function is continuous because:
$$\lim_{x \rightarrow 0+} f(x) = \lim_{x \rightarrow 0-} f(x) = f(0) = 0 $$
But it is not differentiable because:
$$ \lim_{x \rightarrow 0+} f'(x) = 1 $$
But,
$$\lim_{x \rightarrow 0-} f'(x) = \lim_{x \rightarrow 0-} 2x = 0$$
\newpage
% Part
\item (1.5 pts) Find $A'A$ where $$ A=\begin{bmatrix}
	\frac{1}{\sqrt{2}} & -\frac{1}{\sqrt{2}} \\ \frac{1}{\sqrt{2}} & \frac{1}{\sqrt{2}}
\end{bmatrix} $$
Remember: $A'$ is the transpose of $A$.
\[
A'A = \begin{bmatrix}
\frac{1}{\sqrt{2}} & \frac{1}{\sqrt{2}} \\
-\frac{1}{\sqrt{2}} & \frac{1}{\sqrt{2}}
\end{bmatrix}
\begin{bmatrix}
\frac{1}{\sqrt{2}} & -\frac{1}{\sqrt{2}} \\
\frac{1}{\sqrt{2}} & \frac{1}{\sqrt{2}}
\end{bmatrix}
\]
\[
A'A = \begin{bmatrix}
\left( \frac{1}{\sqrt{2}} \times \frac{1}{\sqrt{2}} \right) + \left( \frac{1}{\sqrt{2}} \times \frac{1}{\sqrt{2}} \right) & \left( \frac{1}{\sqrt{2}} \times -\frac{1}{\sqrt{2}} \right) + \left( \frac{1}{\sqrt{2}} \times \frac{1}{\sqrt{2}} \right) \\
\left( -\frac{1}{\sqrt{2}} \times \frac{1}{\sqrt{2}} \right) + \left( \frac{1}{\sqrt{2}} \times \frac{1}{\sqrt{2}} \right) & \left( -\frac{1}{\sqrt{2}} \times -\frac{1}{\sqrt{2}} \right) + \left( \frac{1}{\sqrt{2}} \times \frac{1}{\sqrt{2}} \right)
\end{bmatrix} = \begin{bmatrix}
1 & 0 \\
0 & 1
\end{bmatrix}
\] \\

 % Part
\item (1 pt) Find the derivative of $ f(x) = x^2 \ln x $.
$$ f'(x) = 2x \ln x + \frac{1}{x}\cdot x^2 = x(2 \ln x + 1) $$  \vspace{0.25em}
  % Part
\item (1 pt) Find the derivative of $f(x) = \dfrac{5e^x}{5x+e^x}$
  \[
f'(x) = \frac{(5e^x)(5x + e^x) - (5e^x)(5 + e^x)}{(5x + e^x)^2} = \frac{(5e^x)(5x + e^x - 5 - e^x)}{(5x + e^x)^2} = \frac{25e^x(x - 1)}{(5x + e^x)^2}
\] \vspace{0.25em}
 \item (1 pt) Find the following integral: $ \int e^x dx $
    $$ \int e^x dx =  e^x + c $$
\item (1 pt) Find the following integral: $ \int_0^3 x^2 dx $
$$ \int_0^3 x^2 dx = \left[\frac{x^3}{3}\right]_{0}^3 = \frac{3^3}{3}-\frac{0^3}{3} = 9 $$
\end{enumerate}

%%%%%%%%%%%%%% Question 2
\newpage
\item (8 pts) Two types of cars, gasoline (g) and electric (e), are available in the market. Denote the price of gasoline cars by \(p_g\) and the price of electric cars by \(p_e\). Similarly, denote the quantity of gasoline cars as \(q_g\) and electric cars as \(q_e\). The supply and demand equations for these cars are as follows:

\textit{Supply Equations:}
\begin{align*}
q^s_e &= 25 + 0.3p_e \\
q^s_g &= 50 + 0.3p_g
\end{align*}
\textit{Demand Equations:}
\begin{align*}
q^d_e &= 100 - 0.5p_e + 0.2p_g \\
q^d_g &= 150 + 0.2p_e - 0.5p_g
\end{align*}

\textbf{Note}: At equilibrium, supply is equal to demand: \(q^s_e = q^d_e = q_e\) and \(q^s_g = q^d_g = q_g\).

\begin{enumerate}
\item (2 pts) Write down the \underline{four} equations that must hold in equilibrium. Rearrange each equation so that all terms with variables are on one side and all constants are on the other. What are the four unknown variables?
\begin{align}
q_e-0.3p_e &= 25  \\
q_g - 0.3p_g &= 50  \\
q_e + 0.5p_e - 0.2p_g &= 100  \\
q_g - 0.2p_e + 0.5p_g &= 150 
\end{align}
Four unknown variables: $q_e, p_e, q_g, p_g$. \\
\item (2 pts) Express this system of equations in matrix format as \(Ax = b\). Clearly specify what \(A\), \(x\), and \(b\) are.

\[
\underbrace{\begin{pmatrix}
1 & -0.3 & 0 & 0 \\
0 & 0 & 1 & -0.3 \\
1 & 0.5 & 0 & -0.2 \\
0 & -0.2 & 1 & 0.5
\end{pmatrix}}_{A} 
\underbrace{\begin{pmatrix}
q_e \\
p_e \\
q_g \\
p_g
\end{pmatrix}}_{x}
=
\underbrace{\begin{pmatrix}
25 \\
50 \\
100 \\
150
\end{pmatrix}}_{b}
\]
\item (2 pts) What is the necessary and sufficient condition for a unique solution for this system of equations to exist? 

\textit{Necessary condition:} $A$ must be a square matrix, meaning the number of equations is equal to the number of variables; this condition is met here.

\textit{Sufficient condition:} $A$ must be non-singular (i.e., $|A| \neq 0$), indicating that the rows of $A$ are linearly independent or, equivalently, that each equation is linearly independent. \\

\item (2 pts) How would you solve this system of equations using tools from matrix algebra? There's no need to actually solve it; just explain the steps involved.

To solve this system of equations, we can start by noting that pre-multiplying $Ax =b$ by $A^{-1}$ gives us $x = A^{-1} b$. So we can find the inverse of the matrix $A$ and multiply it with $b$ to find $x$. \\~\\
\end{enumerate}


%%%%%%%%%%%%%% Question 3
\item (5 pts) Consider a utility function for two goods \( x \) and \( y \) given by:
\[
U(x, y) = x^{\frac{1}{3}}y^{\frac{2}{3}}
\]
The marginal utility of each good is given by the partial derivative of the utility function with respect to that good.
\begin{enumerate}
  \item (3 pts) Compute the marginal utility of \( x \) (denoted by \( U_x \)) and the marginal utility of \( y \) (denoted by \( U_y \)) for this utility function.
  \[
  U_x = \frac{\partial U}{\partial x} = \frac{1}{3} x^{-\frac{2}{3}} y^{\frac{2}{3}}
  \]
   \[
  U_y = \frac{\partial U}{\partial y} = \frac{2}{3} x^{\frac{1}{3}} y^{-\frac{1}{3}}
  \]
  \item (2 pts) Is the marginal utility of \( x \) increasing with respect to \( x \)? What about with respect to \( y \)?
  
  $U_x$ is decreasing with respect to \( x \) and increasing with respect to \( y \).
\end{enumerate}

%%%%%%%%%%%%%% Question 4
\newpage
\item (7 pts) Fun with Calculus!
\begin{enumerate}
\item (3.5 pts) Let aggregate wealth \( W \), technology $A$, and population \( N \) be defined as:
\[
W(t) = A(t) + N(t), \quad N(t) = k e^{st}, \quad A(t) = at
\]
where \(k, s,\) and \( a \) are constants, and \( t \) represents time. Determine the total rate of change of aggregate wealth with respect to time. (You need to find the total derivative of $W$ with respect to time.)
\[
\frac{dW}{dt} = \frac{\partial W}{ \partial A}\cdot \frac{dA}{dt} + \frac{\partial W}{\partial N} \cdot \frac{dN}{dt} = 1\cdot a + 1 \cdot sk e^{st} = a + sk e^{st}
\] \vspace{0.25em}
\item (3.5 pts) Denote the demand for a good as a function of its price by $Q(p)$. The following expression holds:
$$ \ln Q(p) =  f(p)  $$
Express the elasticity of demand $Q(p)$ with respect to price in terms of $f'(p)$ and $p$. 

Two ways to approach this:

(1) Take the derivative of the whole equation:
$$ \frac{1}{Q}\cdot \frac{dQ}{dp} = f'(p)$$
Multiply both sides of the equation by $p$. 
$$ \underbrace{\frac{p}{Q}\cdot \frac{dQ}{dp}}_{\varepsilon} = pf'(p)$$

(2) Express $$Q(p) = e^{f(p)}$$
Then,
$$ \varepsilon = \frac{p}{Q}\cdot \frac{dQ}{dp} = \frac{p}{e^{f(p)}}\cdot e^{f(p)} f'(p) = pf'(p) $$
\end{enumerate}
\end{enumerate}

\end{document}