\documentclass{./../../Latex/tests}
\begin{document}
\thispagestyle{plain}
\myheader{Fall/Spring 20XX Midterm Exam}
\rhead{Fall/Spring 20XX Midterm Exam}

\vspace{0.5em}
\testHeader{110}{30}

\section{Short Answer Questions}
\begin{enumerate}
\item The cartesian product of two sets $X$ and $Y$ is defined as: $$ X \times Y = \{(x,y) | x \in X, y \in Y\} $$
What is the cartesian product of $X=\{a,b\}$ and $Y=\{2,1\}$? \\
$$  X \times Y = \{ (a,2), (b,2), (a,1), (b,1) \} $$

\item A matrix's inverse exists if its determinant is equal to 0. 
\begin{itemize}
\item[$\square$] True 
\item[$\text{\rlap{$\checkmark$}}\square$] False \\
\end{itemize}

\item The function $f(x) = |x|$ is \textit{differentiable} at $x=0$.
\begin{itemize}
\item[$\square$] True 
\item[$\text{\rlap{$\checkmark$}}\square$] False \\
\end{itemize}

\item For the function $f(x) = e^x$, $f^{\prime} (x)=f(x)$ 
\begin{itemize}
\item[$\text{\rlap{$\checkmark$}}\square$] True 
\item[$\square$] False \\
\end{itemize}

\item What is the derivative of $y=3x^2+2$? \\  $$ \frac{dy}{dx} =6x$$

\item Consider two sets $A$ and $B$, where $A$ is the set of all odd real numbers and $B$ is the set of all real numbers. What is the intersection of $A$ and $B$? 
$$ A \cap B = A $$

\item Expand the following summation expression: $\sum_{i=0}^{3}(x+i)^{2}$
$$ x^2 + (x+1)^2 + (x+2)^2 + (x+3)^2 $$

\item Find the inverse of $f(x) = \frac{x-2}{3}$.
$$ f^{-1}(y) = 3y+2 \quad \text{or} \quad g(x) =3x+2 $$

\item Why do we need a matrix to be nonsingular when solving systems of linear equations?
\begin{itemize}
\item[$\text{\rlap{$\checkmark$}}\square$] To ensure that the system of equations has a unique solution.
\item[$\square$] To ensure that the system of equations has no solutions.
\item[$\square$] To ensure that the system of equations has infinitely many solutions.
\item[$\square$] It does not matter if the matrix is singular or nonsingular.  \\
\end{itemize}

\item Is the following function continuous? Is it differentiable?
$$ f(x) = \begin{cases}
4 \quad \text{if } x<2 \\
10 \quad \text{if } x\geq2
\end{cases} $$
Neither continuous, nor differentiable. \\

\item For the function $f(x) = \ln x$, $f^{\prime} (x)=1/x$ 
\begin{itemize}
\item[$\text{\rlap{$\checkmark$}}\square$] True 
\item[$\square$] False \\
\end{itemize}  
\item Find the derivative of $y = \frac{1}{x}$.
$$ \frac{dy}{dx} =-\frac{1}{x^2} $$
\item Find the derivative of $y = (2-3x)(1+x)$.
$$ \frac{dy}{dx} =-3(1+x)+1(2-3x) = -3-3x+2-3x = -(1+6x) $$
\end{enumerate}

\newpage
\section{Linear Algebra}

\begin{enumerate}
\item (5 pts) Given the vector $x$ and matrix $A$ below, show that $x'Ax$ represents a weighted sum of squares. What is the dimension of $x'Ax$?
$$
x = \begin{bmatrix}
x_1 \\ x_2
\end{bmatrix} \quad \quad \quad
A = \begin{bmatrix}
a_{11} & 0 \\ 0 & a_{22}
\end{bmatrix} \quad \quad
$$
$$
\begin{aligned}
& x^{\prime} A x=\left[\begin{array}{ll}x_{1} & x_{2}\end{array}\right]_{1 \times 2}\left[\begin{array}{cc}a_{11} & 0 \\0 & a_{22}\end{array}\right]_{2 \times 2}\left[\begin{array}{l}x_{1} \\x_{2}\end{array}\right]_{2 \times 1} \\
& =\left[\begin{array}{lll}a_{11} x_{1} & a_{22} x_{2}\end{array}\right]_{1 \times 2}\left[\begin{array}{l}x_{1} \\x_{2}\end{array}\right]_{2 \times 1} \\
& =a_{11} x_{1}^{2}+a_{22} x_{2}^{2}=\sum_{i=1}^{2} a_{i i} x_{i}{ }^{2}
\end{aligned}
$$
Dimension of $x'Ax$ is $1 \times 1$.

\item (4 pts) Say I have a system of $m$ equations with $n$ unknowns. 
\begin{align*}
	a_{11} x_1 + a_{12} x_2 + \cdots a_{1n} x_n &= b_1 \\
	a_{21} x_1 + a_{22} x_2 + \cdots a_{2n} x_n &= b_2 \\
	\vdots \quad \quad \quad \quad \quad \quad \quad \vdots \\
	a_{m1} x_1 + a_{m2} x_2 + \cdots a_{mn} x_n &= b_m \\
\end{align*}
\begin{enumerate}
	\item (1 pt) What is the necessary condition for the existence of a unique solution for this system in terms of $m$ and $n$? 
	\item [] \textit{Necessary condition for the existence of a unique solution is that the number of equations is equal to the number of unknowns i.e. $m=n$. } \\
	\item (1 pt) What is the sufficient condition for the existence of a unique solution for this system? 
	\item [] \textit{Sufficient condition for the existence of a unique solution is that all the equations are linearly independent.} \\
	\item (2 pts) How would you use the tools learned in linear algebra to solve this system of equations? 
	\item [] I \textit{would start by writing out the above system of equations in matrix format, i.e. $$ Ax = b $$
	where }
$$
A = \begin{bmatrix}
a_{11} & a_{12} & \hdots & a_{1n} \\
a_{21} & a_{22} & \hdots & a_{2n} \\
\vdots & \vdots & \vdots & \vdots \\
a_{n1} & a_{n2} & \hdots & a_{nn} \\
\end{bmatrix}, \quad  
x = \begin{bmatrix} x_1 \\x_2 \\ \vdots \\x_n \end{bmatrix} , \quad  
b = \begin{bmatrix} b_1 \\b_2 \\ \vdots \\b_n \end{bmatrix} 
$$
\textit{Now note that premultiplying $Ax = b$ by $A^{-1}$ implies that $x=A^{-1}b$. So I would find the inverse of $A$ and multiply it with the vector $b$ to find the solution to this system of equations.
}
\end{enumerate}

\item (10 pts) Consider the following system of equations:
\begin{align*}
4 x + 3y -2z & = 7 \\
x+y & = 5 \\
3x+z &= 4	
\end{align*}
\begin{enumerate}
% Part (a)
  \item (1.5 pt) Write this system of equations in matrix format, i.e., $$ Av=b $$
  What is $A$, $v$, and $b$ equal to?
$$ A = \begin{bmatrix}
	4 & 3 & -2 \\
	1 & 1 & 0 \\
	3 & 0 & 1 
\end{bmatrix} \quad \quad
v = \begin{bmatrix}
	x \\ y \\ z
\end{bmatrix} \quad \quad 
b = \begin{bmatrix}
	7 \\ 5 \\ 4
\end{bmatrix}$$ 
% Part (b)
   \item (3 pts) Calculate the adjoint of $A$. \\
    
To find the adjoint of a matrix, we need to find the transpose of the matrix of cofactors. Let's first find the 9 cofactors. 

$$\begin{aligned}
	& C_{11} = (-1)^2 \begin{vmatrix}
		1 & 0 \\
		0 & 1
	\end{vmatrix} =1  &
	& C_{12} = (-1)^3 \begin{vmatrix}
		1 & 0 \\
		3 & 1
	\end{vmatrix} =-1  &
	& C_{13} = (-1)^4 \begin{vmatrix}
		1 & 1 \\
		3 & 0
	\end{vmatrix} =-3 \\~\\
	& C_{21} = (-1)^3 \begin{vmatrix}
		3 & -2 \\
		0 & 1
	\end{vmatrix} =-3  &
	& C_{22} = (-1)^4 \begin{vmatrix}
		4 & -2 \\
		3 & 1
	\end{vmatrix} =10  &
	& C_{23} = (-1)^5 \begin{vmatrix}
		4 & 3 \\
		3 & 0
	\end{vmatrix} =9 \\~\\
	& C_{31} = (-1)^4 \begin{vmatrix}
		3 & -2 \\
		1 & 0
	\end{vmatrix} =2  &
	& C_{32} = (-1)^5 \begin{vmatrix}
		4 & -2 \\
		1 & 0
	\end{vmatrix} =-2  &
	& C_{33} = (-1)^6 \begin{vmatrix}
		4 & 3 \\
		1 & 1
	\end{vmatrix} =1 \\
\end{aligned}$$ \\

Then the adjoint of $A$ is given by:
$$ Adj A = C' = \begin{bmatrix}
	1 & -3 & 2 \\
	-1 & 10 & -2 \\
	-3 & 9 & 1 
\end{bmatrix} $$

\item (2 pts) Calculate the determinant of $A$. Is $A$ nonsingular? \\
 
To calculate the determinant of $A$ by expanding with respect to the third row:
$$ |A| = a_{31}|C_{31}| + a_{32}|C_{32}| + a_{33}|C_{33}| = 3.2+0.(-2)+1.1 = 7 $$
Since $|A| \neq 0$, $A$ is nonsingular. \\
  \item (1.5 pt) If you premultiply $A^{-1}$ on both sides of the equation $ Av=b $, you should be able to derive an expression to solve for $v$. Write down this expression. 
  $$ \underbrace{A^{-1}A}_{I}v=A^{-1}b \rightarrow v^* =A^{-1}b  $$\\
  
  \item (2 pts) Using the expression in $(d)$ solve for $v^*$. \\
  
  Note that, $$ A^{-1} = \frac{1}{|A|} Adj A$$
  Then, 
  $$ v^* =A^{-1}b = \frac{1}{7}\begin{bmatrix}
	1 & -3 & 2 \\
	-1 & 10 & -2 \\
	-3 & 9 & 1 
\end{bmatrix} \begin{bmatrix}
	7 \\ 5 \\ 4
\end{bmatrix} = \frac{1}{7} \begin{bmatrix}
	7-15+8 \\ -7+50-8 \\ -21+45+4
\end{bmatrix}  =\frac{1}{7} \begin{bmatrix}
	0 \\ 35 \\ 28
\end{bmatrix} = \begin{bmatrix}
	0 \\ 5 \\ 4
\end{bmatrix}$$
\end{enumerate}

\end{enumerate}

\newpage
\section{Calculus}

\begin{enumerate}

\item (6 pts) Find the derivative for the following functions (2 pts each): \\
\begin{enumerate}
\item $y=\ln(x^2+1)$ \\ 
$$
\frac{d y}{d x}=\frac{2 x}{x^2+1}
$$ 
\item $y=\dfrac{e^x}{1+e^x}$ \\ 
$$\frac{d y}{d x} =\frac{e^x\left(1+e^x\right)-e^x e^x}{\left(1+e^x\right)^2} =\frac{e^x}{\left(1+e^x\right)^2} $$ 
\item $y=v+v^3 \quad \text{ where } v=x+1 $ \\ 
$$\frac{d y}{d x} =\frac{d y}{d v} \cdot \frac{d v}{d x} =\left(1+3 v^2\right) \cdot 1 =1+3(x+1)^2 $$ \\
\end{enumerate} 

\item (5 pts) Given the consumption function $$ C= 200+0.6Y  $$
where $C$ is consumption, and $Y$ is income. 
\begin{enumerate}
  \item (3 pts) Find the income elasticity of consumption $\varepsilon_{CY}$, and determine its sign, assuming $Y>0$. \\ 
 $$
\varepsilon_{CY}=\frac{d C}{d Y} \cdot \frac{Y}{C}=\frac{0.6 Y}{200+0.6 Y}>0 
$$ \\

  \item (1 pt) Show that this consumption function is inelastic at all positive income levels. 
  $$ 0.6 Y<200+0.6 Y \rightarrow \varepsilon_{CY}<1 $$ \\
  \item (1 pt) What is the income elasticity of consumption when income is equal to \$1000? 
$$
\varepsilon_{CY}=\frac{0.6 \times 1000}{200+0.6 \times 1000}=\frac{600}{800}=\frac{3}{4}=0.75
$$ 
  \item (1 pt) If income increases by 1\% from \$1000 to \$1010, by what percent does consumption increase? \\
  By the definition of elasticity, a 1\% increase in income leads to a 0.75\% increase in consumption. \\
\end{enumerate}

\item (5 pts) Given the following function:
$$f(x,y,z) = xyz$$ 
\begin{enumerate}
\item (2 pts)  Find the partial derivatives $f_x, f_y$, and $f_z$. 
$$f_x=y z, f_y=x z, f_z=x y$$ 
\item (1 pt) Find the gradient of $f$. 
$$\nabla f=\left[\begin{array}{l}y z \\ x z \\ x y\end{array}\right]$$ 
\item (1 pt) Find the total differential of $f$. You can denote it by $d f$. 
$$\begin{aligned} d f &=f_x \cdot d x+f_y \cdot d y+f_z \cdot d z \\ &=y z \cdot d x+x z \cdot d y+x y \cdot d z
\end{aligned}$$	
\item (1 pt) Find the total derivative of $f$ with respect to $x$? 
$$  \frac{d f}{d x} =f_x+f_y \cdot \frac{d y}{d x}+f_z \cdot \frac{d z}{d x} $$
Since, $ \frac{d y}{d x}=0, \frac{d z}{d x}=0$ 
$$\frac{d f}{d x} =f_x=y z $$
\end{enumerate}

\item (6 pts) Fun with Calculus!
\begin{enumerate}
\item (3 pts) Demand for a good as a function of its price is given as follows:
$$ Q(p) = p^{-\frac{1}{1+\alpha}}  $$
Calculate the elasticity of demand with respect to price. (Note: You can also take the log of both sides of the equation and write $\ln Q = -\frac{1}{1+\alpha} \cdot \ln p$, and use that equation if you like.) \\~\\

Note that,
$$ \frac{dQ}{dp} = -\frac{1}{1+\alpha}\cdot p^{-\frac{1}{1+\alpha}-1}$$
Plugging this and $Q = p^{-\frac{1}{1+\alpha}}$ in the formula for elasticity:
$$ \varepsilon = \frac{dQ}{dp}\cdot \frac{p}{Q} = -\frac{1}{1+\alpha}\cdot p^{-\frac{1}{1+\alpha}-1} \cdot \frac{p}{p^{-\frac{1}{1+\alpha}}} = -\frac{1}{1+\alpha}  $$ \\

I think my note in the parenthesis confused some of you. I was suggesting that alternatively you could write the equation in logs and find the elasticity as follows:
$$\ln Q = -\frac{1}{1+\alpha} \cdot \ln p$$
Differentiating both sides with respect to $p$:
$$ \frac{dQ}{dp}\cdot\frac{1}{Q} = -\frac{1}{1+\alpha} \cdot \frac{1}{p}  $$
Rearrange above equation to bring the $p$ on the left-hand side:
$$ \frac{dQ}{dp}\cdot\frac{p}{Q} = -\frac{1}{1+\alpha} =\varepsilon  $$

\item (3 pts) Suppose that aggregate income $Y$ and population $P$ are given by:
$$Y(t) = \ln P(t), \quad \quad P(t) = ae^{rt}$$ 
where $c, a$, and $r$ are constants. $t$ denotes time. Find the growth rate of income, which is given by the derivative of $Y$ with respect to $t$. \\

We can find this using the chain rule:
$$ \frac{dY}{dt} =  \frac{dY}{dP} \cdot \frac{dP}{dt} = \frac{1}{P(t)}\cdot are^{rt}= \frac{1}{ae^{rt}}\cdot are^{rt} = r $$
In the last step we are just plugging in $P(t) = ae^{rt}$.
\end{enumerate}


\item (6 pts) Consider the following production function with two inputs, capital ($K$) and labor ($L$):
$$ Q = 2K^{1/2}L^{1/2} $$
The marginal product of an input is given by the partial derivative of the production function with respect to that input variable. 
\begin{enumerate}
\item (3 pts) Show that the marginal product of capital (MPK) and labor (MPL)for the above production function are given by:
$$ MPK = \frac{1}{2}\cdot\frac{Q}{K} \quad \quad MPL = \frac{1}{2}\cdot \frac{Q}{L} $$ \\
MPK is the partial derivative of $Q$ with respect to $K$:
$$ MPK = \frac{\partial Q}{\partial K} = K^{-1/2} L^{1/2}  $$
MPL is the partial derivative of $Q$ with respect to $L$:
$$ MPL = \frac{\partial Q}{\partial L} = K^{1/2} L^{-1/2} $$
To see that the expressions given in the question are the same as the partial derivatives above, we can plug-in $Q=2K^{1/2}L^{1/2}$ in both expressions as follows:
$$ MPK = \frac{1}{2}\cdot\frac{Q}{K}=  \frac{1}{2}\cdot\frac{2K^{1/2}L^{1/2}}{K} =  K^{-1/2} L^{1/2} $$
$$ MPL = \frac{1}{2}\cdot\frac{Q}{L} = \frac{1}{2}\cdot\frac{2K^{1/2}L^{1/2}}{L} =  K^{1/2} L^{-1/2} $$ \\

\item (2 pts) Now, say that in equilibrium, wages ($w$) are equal to the marginal product of labor i.e.
$$ w = \frac{1}{2}\cdot \frac{Q}{L} = K^{1/2}L^{-1/2}   $$
Given $K=100$, write labor demand $L$ as a function of wages $w$. (Essentially, you are finding the inverse of a function). \\

With $K=100$, we have:
$$ w = (100)^{1/2} L^{-1/2} \rightarrow  L = \frac{100}{w^2}$$ 

\item (1 pt) Given your answer in (b), do you think labor demand increases or decreases with an increase in wages? \\
Labor demand decreases with an increase in wages.
\end{enumerate}


\end{enumerate}


\end{document}