\documentclass{./../../Latex/handout}
\begin{document}
\thispagestyle{plain}
\newcommand{\mytitle}{Sets and Functions}
\myheader{\mytitle}

\vspace{0.5em}
Distributive law
$$ A \cup (B \cap C) = (A \cup B) \cap (A \cup C) $$
$$ A \cap (B \cup C) = (A \cap B) \cup (A \cap C) $$

\vspace{1em}

Verify the distributive law for:
$$ A = \{1,2,3\}, B = \{2,4,6\}, C=\{4,8\} $$

\underline{First part} \\~\\
Left hand side: $ A \cup (B \cap C) = $  \\~\\
\\~\\
Right hand side: $ (A \cup B) \cap (A \cup C) = $ \\~\\
\\~\\
\underline{Second part} \\~\\
Left hand side: $ A \cap (B \cup C) = $  \\~\\
\\~\\
Right hand side: $ (A \cap B) \cup (A \cap C) = $ \\~\\

\newpage
\underline{Definitions}:
\begin{witemize}
	\item A \textit{function} $y=f(x)$ is a relation where for each $x$ there is a unique $y$. (One input does not give multiple outputs.)
	\item For a \textit{one-to-one function}, each value of $y$ is associated with a unique value of $x$. (Different inputs lead to different outputs.)
	\item \textit{Inverse of a function} $x=f^{-1}(y)$ returns the corresponding value of $x$ for each $y$. 
	\item Only one-to-one functions have an inverse.
	\item Only strictly monotonic functions are one-to-one.
\end{witemize}
\vspace{1cm}
\underline{Questions}:
\begin{itemize}
	\item Is $f$ a function if for $ x_1 \neq x_2 $, $f(x_1) = f(x_2)$? If yes, is it a one-to-one function? \\ \vspace{3cm}
	\item Consider the function $g: \mathbb{R}_{+} \rightarrow \mathbb{R}$ such that $g(x) = x^2 + 4 $. Is $g$ a strictly increasing function? Find the inverse of $g$. 
\end{itemize}




\end{document}