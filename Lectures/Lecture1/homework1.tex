\documentclass{./../../Latex/handout}
\begin{document}

\thispagestyle{plain}
\newcommand{\mytitle}{Homework 1}
\myheader{\mytitle}

%%%%%%%%%%%%%% Exercise $2.3$
\underline{Exercise $2.3$} 
\begin{enumerate}

\item Write the following in set notation:
\begin{enumerate}
	\item The set of all real numbers greater than 34.
	\item The set of all real numbers greater than 8 but less than 65.
\end{enumerate}
 
\item Given the sets $S_1=\left\{2,4, 6\}, S_2=\{7,2,6\}, S_3=\{4,2,6\}\right.$, and $S_4=\{2,4\}$, which of the following statements are true?
\begin{tasks}(3)
\task $S_1=S_3$
\task $S_1=\mathbb{R}$ 
\task $8 \in S_2$
\task $3 \notin S_2$
\task $4 \notin S_3$
\task $S_4 \subset\mathbb{R}$
\task $ S_1 \supset S_4$
\task $ \emptyset \subset S_2$
\task $S_3 \supset \{1,2\}$ 
\end{tasks}
Note that $\mathbb{R}$ denotes the set of real numbers.
\end{enumerate}


\underline{Exercise $2.4$} 
\begin{enumerate}
\item[5.] If the domain of the function $y=5+3x$ is the set $\{x | 1 \leq x \leq 9 \}$, find the range of the function and express it as a set.
\item[7.] In the theory of the firm, economists consider the total cost $C$ to be a function of the output level $Q$: $C=f(Q)$.
\begin{enumerate}
\item According to the definition of a function, should each cost figure be associated with a unique output level?
\item Should each level of output determine a unique cost figure?
\end{enumerate}
\item[8.] If an output level $Q_1$ can be produced at a cost of $C_1$, then it must also be possible (by being less efficient) to produce $Q_1$ at a cost of $C_1+\$ 1$, or $C_1+\$ 2$, and so on. Thus it would seem that output $Q$ does not uniquely determine total cost $C$. If so, to write $C=f(Q)$ would violate the definition of a function. How, in spite of this reasoning, would you justify the use of the function $C=f(Q)$? 
\end{enumerate}

\underline{Exercise $2.5$} 
\begin{enumerate}
\item Graph the following functions and find their inverse functions.
\begin{enumerate}
	\item $y = 16 + 2x$
	\item $y = 8-2x$
	\item $y = 2x+12 $ 
\end{enumerate}
\end{enumerate}

\underline{Exercise $4.2$} 

\begin{enumerate}
% Question 6
\item[6.]  Expand the following summation expressions:
\begin{tasks}(3)
\task[(a)] $\sum_{i=2}^5 x_i$
\task[(b)] $\sum_{i=5}^8 a_i x_i$
\task[(c)] $\sum_{i=1}^4 b x_i$
\task[(d)] $\sum_{i=1}^n a_i x^{i-1}$
\task[(e)] $\sum_{i=0}^3(x+i)^2$ \\
\end{tasks}

% Question 8
\item[8.] Show that the following are true:
\begin{enumerate}
\item[(a)] $\begin{aligned} \left(\sum_{i=0}^n x_i\right)+x_{n+1}=\sum_{i=0}^{n+1} x_i \end{aligned}$
\item[(b)] $\begin{aligned}\sum_{j=1}^n a b_j y_j=a \sum_{j=1}^n b_j y_j\end{aligned}$
\item[(c)] $\begin{aligned}\sum_{j=1}^n\left(x_j+y_j\right)=\sum_{j=1}^n x_j+\sum_{j=1}^n y_j\end{aligned}$ \\
\end{enumerate}
\end{enumerate}

%%%%%%%%%%%%%% Exercise 5.1
\underline{Exercise 5.1}
\begin{enumerate}
\item[1.] In the following paired statements, let $p$ be the first statement and $q$ the second. Which is true for each case: $p \Rightarrow q, p \Leftarrow q$, or $p \Leftrightarrow q$ ?
\begin{enumerate}
\item It is a holiday; it is Thanksgiving Day.
\item A geometric figure has four sides; it is a rectangle.
\item Two ordered pairs $(a, b)$ and $(b, a)$ are equal; $a$ is equal to $b$.
\item A number is rational; it can be expressed as a ratio of two integers.
\item A $4 \times 4$ matrix is nonsingular; the rank of the $4 \times 4$ matrix is 4. (\textit{skip for now})
\item The gasoline tank in my car is empty; I cannot start my car.
\item The letter is returned to the sender with the marking ``addressee unknown''; the sender wrote the wrong address on the envelope.\end{enumerate}
\end{enumerate}

\end{document}