\documentclass{./../../Latex/handout}
\begin{document}
\thispagestyle{plain}
\newcommand{\mytitle}{Homework 1 Solutions}
\myheader{\mytitle}

%%%%%%%%%%%%%%%%%%%%%%%%%%%%%%%%% Exercise 2.3
\underline{Exercise $2.3$} 

\begin{enumerate}
\item \begin{enumerate}
	\item $ \{x \in \mathbb{R} \mid x>34\}$ or  $\{x \mid x>34\} $
	\item $\{x \mid 8<x<65\}$ 
\end{enumerate}
\item (a), (d), (f), (g), and (h) are true. 
\end{enumerate}

%%%%%%%%%%%%%%%%%%%%%%%%%%%%%%%%% Exercise 2.4
\underline{Exercise $2.4$} 

\begin{enumerate}
\item[5.] We are given the function $y=5+3x$ with domain $X = \{x | 1 \leq x \leq 9 \}$. Note that for this function, when $x=1$, $y=8$ and when $x=9$, $y=32$. So the range for this function is:
$$
f(X) =\{y \mid 8 \leq y \leq 32\}
$$
Note: It is not always the case that extreme values of the domain correspond to extreme values of the range. For example, consider $y=x^{2}$ with domain $\{x \mid-2 \leq x \leq 2\}$, the range here is $\{y \mid 0 \leq y \leq 4\}$.

\item[7.] (a) No, (b) Yes
\item[8.] For each output, we would want to produce at the lowest cost. \\
\end{enumerate}

%%%%%%%%%%%%%%%%%%%%%%%%%%%%%%%%% Exercise 2.5
\newpage
\underline{Exercise $2.5$} 
\begin{enumerate}
	
\item Graph the following functions and find their inverse.
\definecolor{mycolor}{RGB}{145,0,0}

\begin{tasks}(2)
\task $\begin{aligned} y = 16 + 2x, \quad  f^{-1}(y) = \frac{y-16}{2} \end{aligned}$ \\~\\
\begin{tikzpicture}[scale=0.85]
\begin{axis}[axis lines = center, xlabel = \(x\), ylabel = \(y\), ytick distance=8, xtick distance=4, extra x ticks = {0}, xlabel style={at={(axis description cs:0.8,1.05)}, anchor=north}, ylabel style={at={(axis description cs:1.05,0.3)}, anchor=north}]
\addplot [domain=-9:1, samples=100, color=mycolor, line width = 0.4mm]
{16 + 2*x};
\end{axis}
\end{tikzpicture} 

\task $ \begin{aligned} y = 8-2x \quad  f^{-1}(y) = \frac{8-y}{2}\end{aligned}$ \\~\\
\begin{tikzpicture}[scale=0.85]
\begin{axis}[axis lines = center, xlabel = \(x\), ylabel = \(y\), ytick distance=4, xtick distance=2, extra x ticks = {0}, xlabel style={at={(axis description cs:0.25,1.05)}, anchor=north}, ylabel style={at={(axis description cs:1.05,0.4)}, anchor=north}]
\addplot [domain=-1:5, samples=100, color=mycolor, line width = 0.4mm]
{8-2*x};
\end{axis}
\end{tikzpicture} 

\task $\begin{aligned} y = 2x+12 \quad  f^{-1}(y) = \frac{y-12}{2}\end{aligned}$ \\~\\
\begin{tikzpicture}[scale=0.85]
\begin{axis}[axis lines = center, xlabel = \(x\), ylabel = \(y\), ytick distance=4, xtick distance=2, extra x ticks = {0}, xlabel style={at={(axis description cs:0.75,1.05)}, anchor=north}, ylabel style={at={(axis description cs:1.05,0.3)}, anchor=north}]
\addplot [domain=-8:2, samples=100, color=mycolor, line width = 0.4mm]
{2*x+12};
\end{axis}
\end{tikzpicture}
\end{tasks}
\end{enumerate}

%%%%%%%%%%%%%%%%%%%%%%%%%%%%%%%%% Exercise 4.2
\underline{Exercise 4.2}  
% Question 6
\begin{enumerate}
\item[6.]
\begin{tasks}(2)
\task  $x_{2}+x_{3}+x_{4}+x_{5}$
\task $a_{5} x_{5}+a_{6} x_{6}+a_{7} x_{7}+a_{8} x_{8}$
\task $b x_{1}+b x_{2}+b x_{3}+b x_{4}$
\task $a_{1}+a_{2} x+a_{3} x^{2}+\ldots+a_{n} x^{n-1}$
\task $x^{2}+(x+1)^{2}+(x+2)^{2}+(x+3)^{2}$  \\~\\
\end{tasks}

% Question 8
\item[8.] 
\begin{enumerate}
\item $$ \left(\sum_{i=0}^{n} x_{i} \right)+x_{n+1} =x_{0}+x_{1}+x_{2}+\ldots+x_{n+1}  =\sum_{i=0}^{n+1} x_{i} $$
\item $$ \begin{aligned} 
\sum_{j=1}^{n} a b_{j} y_{j}&=a b_{1} y_{1}+a b_{2} y_{2}+\ldots+a b_{n} y_{n} \\ 
&= a\left(b_{1} y_{1}+b_{2} y_{2}+\ldots+b_{n} y_{n}\right) \\
&= a \sum_{j=1}^{n} b_{j} y_{j} \end{aligned}$$
\item $$ \begin{aligned}  \sum_{j=1}^{n}\left(x_{j}+y_{j}\right) 
& =\left(x_{1}+y_{1}\right)+\left(x_{2}+y_{2}\right)+\ldots+\left(x_{n}+y_{n}\right) \\
& =x_{1}+x_{2}+\ldots+x_{n}+x_{1}+y_{2}+\ldots+y_{n} \\
& =\sum_{j=1}^{n} x_{j}+\sum_{j=1}^{n} y_{j} \end{aligned} $$
\end{enumerate}
\end{enumerate}

%%%%%%%%%% Exercise 5.1
\underline{Exercise 5.1}  
\begin{enumerate}
\item[1.] \begin{tasks}(3)
\task $ q \implies p $
\task $ q \implies p $
\task $ q \iff p $
\task $ q \iff p $
\task $ q \iff p $
\task $ p \implies q $
\task $ q \implies p $ 
\end{tasks}
\end{enumerate}



\end{document}