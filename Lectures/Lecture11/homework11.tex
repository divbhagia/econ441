\documentclass{./../../Latex/homework}
\begin{document}
\thispagestyle{plain}
\newcommand{\mytitle}{Homework 11 Problems}
\myheader{\mytitle}

%%%%%%%%%%%%%% Exercise 11.5
\subsection*{Exercise 11.5}

For questions 1 (a) and 2 (c), use the following definitions to check whether the function is concave, convex, strictly concave, or strictly convex.

For any two distinct points $u$ and $v$ and $0<\lambda<1$,
\begin{align*}
	f(\lambda u + (1-\lambda) v) \geq \lambda f(u) + (1-\lambda) f(v) \quad & \rightarrow \quad f(.) \text{ is concave.} \\
	f(\lambda u + (1-\lambda) v) > \lambda f(u) + (1-\lambda) f(v) \quad & \rightarrow \quad f(.) \text{ is strictly concave.} \\
	f(\lambda u + (1-\lambda) v) \leq \lambda f(u) + (1-\lambda) f(v) \quad & \rightarrow \quad f(.) \text{ is convex.} \\
	f(\lambda u + (1-\lambda) v) < \lambda f(u) + (1-\lambda) f(v) \quad & \rightarrow \quad f(.) \text{ is strictly convex.} 
\end{align*}

\begin{enumerate}

% Question 1 (a)
\item \begin{enumerate} \item  $z=x^{2}$ 
\end{enumerate}


% Question 2 (c)
\item \begin{enumerate} \item[(c)]  $z=xy$ 
\end{enumerate}

% Question 4
\item[4.] Do the following constitute convex sets in the 3D space?
\begin{tasks}(3)
\task A doughnut
\task A bowling pin
\task A perfect marble
\end{tasks}

% Question 5
\item[5.] The equation $x^{2}+y^{2}=4$ represents a circle with center at $(0,0)$ and with a radius of 2 .
\begin{enumerate} 
\item Interpret geometrically the set $\left\{(x, y) \mid x^{2}+y^{2} \leq 4\right\}$.
\item Is this set convex?
\end{enumerate}

\end{enumerate}

%%%%%%%%%%%%%% Exercise 12.4
\subsection*{Exercise 12.4}
For questions 1 and 2, use the following definitions to conclude whether a function is (strictly) quasiconcave or (strictly) quasiconvex. 

Given two distinct points $u$ and $v$, if $f(v) \geq f(u)$ then for any $0 <\lambda< 1 $, 
\begin{align*}
	f(\lambda u + (1-\lambda) v) \geq f(u) \quad & \rightarrow \quad f(.) \text{ is quasiconcave.} \\
	f(\lambda u + (1-\lambda) v) > f(u) \quad & \rightarrow \quad f(.) \text{ is strictly quasiconcave.} \\
	f(\lambda u + (1-\lambda) v) \leq f(v) \quad & \rightarrow \quad f(.) \text{ is quasiconvex.} \\
	f(\lambda u + (1-\lambda) v) < f(v) \quad & \rightarrow \quad f(.) \text{ is strictly quasiconvex.} 
\end{align*}

\begin{enumerate}

% Question 1
\item Draw a strictly quasiconcave curve $z=f(x)$ which is
\begin{tasks}(2)
\task also quasiconvex
\task not quasiconvex
\task not convex
\task not concave
\task neither concave nor convex
\task both concave and convex 	
\end{tasks}


% Question 2
\item Are the following functions quasiconcave? Strictly so? Assume that $x \geq 0$.
\begin{tasks}
\task $f(x)=a$
\task $f(x)=a+b x \quad (b>0)$
\task $f(x)=a+c x^{2} \quad (c<0)$
\end{tasks}
\end{enumerate}

For question 4, use the following alternate definitions: 
\begin{itemize}
  \item A function $f(x)$, where $x$ is a vector of variables, is (strictly) quasiconcave iff for any constant $k$, the upper-contour set
$ S^U = \{x | f(x) \geq k \} $ is a (strictly) convex set. 
\item Similarly, a function is (strictly) quasiconvex iff for any constant $k$, the lower-contour set $ S^L = \{x | f(x) \leq k \} $ is a (strictly) convex set.
\end{itemize}

\begin{enumerate}
% Question 4
\item[4.] Check whether the following functions are quasiconcave, quasiconvex, both, or neither:
\begin{tasks}
\task $f(x)=x^{3}-2 x$
\task $f\left(x_{1}, x_{2}\right)=6 x_{1}-9 x_{2}$
\task $f\left(x_{1}, x_{2}\right)=x_{2}-\operatorname{in} x_{1}$
\end{tasks}

\end{enumerate}

%%%%%%%%%%%%%% Exercise 12.4
\subsection*{Exercise 12.6}

\begin{enumerate}

% Question 1
\item Determine whether the following functions are homogeneous. If so, of what degree?
\begin{tasks}(3)
\task $f(x, y)=\sqrt{x y}$
\task $f(x, y)=\left(x^2-y^2\right)^{1 / 2}$
\task $f(x, y)=x^3-x y+y^3$
\task $f(x, y)=2 x+y+3 \sqrt{x y}$
\task $f(x, y, w)=\frac{x y^2}{w}+2 x w$
\task $f(x, y, w)=x^4-5 y w^3$
\end{tasks}

% Question 2
\item Show that a production function $Q=f(K, L)$ that is homogenous of degree 1 can be written as $Q=K\psi\left(\frac{L}{K}\right)$ and $Q=L \phi\left(\frac{K}{L}\right)$.

% Question 6
\item[6.] Given the production function $Q=A K^{\alpha} L^{\beta}$, show that:
\begin{enumerate}
\item $\alpha+\beta>1$ implies increasing returns to scale.
\item $\alpha+\beta<1$ implies decreasing returns to scale.
\item $\alpha$ and $\beta$ are, respectively, the partial elasticities of output with respect to the capital and labor inputs.
\end{enumerate}

% Question 7
\item[7.] Let output be a function of three inputs: $Q=A K^a L^b N^c$.
\begin{enumerate}
\item Is this function homogeneous? If so, of what degree?
\item Under what condition would there be constant returns to scale? Increasing returns to scale?
\item Find the share of product for input $N$, if it is paid by the amount of its marginal product.
\end{enumerate}

\end{enumerate}


\end{document}