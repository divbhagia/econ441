\documentclass{./../../Latex/handout}
\begin{document}
\thispagestyle{plain}
\newcommand{\mytitle}{Homework 2 Problems}
\myheader{\mytitle}

%%%%% Exercise $4.2$
\underline{Exercise $4.2$}
\begin{itemize}

% Question 1
\item[1.] Given $A=\left[\begin{array}{cc}7 & -1 \\ 6 & 9\end{array}\right],  B=\left[\begin{array}{cc}0 & 4 \\ 3 & -2\end{array}\right], \text{ and }C=\left[\begin{array}{ll}8 & 3 \\ 6 & 1\end{array}\right]$, find: 
\begin{tasks}(2)
	\task[(a)] $A+B$
	\task[(b)] $C-A$
	\task[(c)] $3A$
	\task[(d)] $4B+2C$ 
\end{tasks} 

% Question 2
\item[2.] Given $A = \left[\begin{array}{cc}
2 & 8 \\
3 & 0 \\
5 & 1
\end{array}\right], B=\left[\begin{array}{cc} 2 & 0 \\ 3 & 8\end{array}\right], \text{ and }
C=\left[\begin{array}{cc} 7 & 2 \\ 6 & 3\end{array}\right]$:
\begin{enumerate}
	\item[(a)] Is $AB$ defined? Calculate $AB$. Can you calculate $BA$? Why?
	\item[(b)] Is $BC$ defined? Calculate $BC$. Is $CB$ defined? If so, calculate $CB$. Is it true that $BC=CB$. 
\end{enumerate}

% Question 4
\item[4.] Find the product matrices in the following (in each case, append beneath every matrix a dimension indicator): 
\begin{tasks}(2)
\task[(a)] $\left[\begin{array}{lll}0 & 2 & 0 \\ 3 & 0 & 4 \\ 2 & 3 & 0\end{array}\right]\left[\begin{array}{ll}8 & 0 \\ 0 & 1 \\ 3 & 5\end{array}\right]$
\task[(b)] $\left[\begin{array}{rrr}6 & 5 & -1 \\ 1 & 0 & 4\end{array}\right]\left[\begin{array}{rr}4 & -1 \\ 5 & 2 \\ 0 & 1\end{array}\right]$
\task[(c)] $\left[\begin{array}{rrr}3 & 5 & 0 \\ 4 & 2 & -7\end{array}\right]\left[\begin{array}{l}x \\ y \\ z\end{array}\right]$
\task[(d)] $\left[\begin{array}{lll}a & b & c\end{array}\right]\left[\begin{array}{ll}7 & 0 \\ 0 & 2 \\ 1 & 4\end{array}\right]$ 
\end{tasks}
\end{itemize}

%%%%% Exercise $4.4$
\underline{Exercise $4.4$}
\begin{itemize}
	
% Question 5
\item[5.] 
\begin{itemize}
\item[(e)] Find (i) $C=A B$, and (ii) $D=B A$, if
$$
A=\left[\begin{array}{r}
-2 \\
4 \\
7
\end{array}\right] \quad B=\left[\begin{array}{lll}
3 & 6 & -2
\end{array}\right]
$$
\end{itemize}

% Question 7
\item[7.] If the matrix $A$ in Example 5 had all its four elements nonzero, would $x^{\prime} A x$ still give a weighted sum of squares? Would the associative law still apply? \\
\end{itemize}

%%%%% Exercise $4.5$
\underline{Exercise $4.5$}
\begin{itemize}
% Question 1
\item[1.] Given $A=\left[\begin{array}{rrr}-1 & 5 & 7 \\ 0 & -2 & 4\end{array}\right], b=\left[\begin{array}{l}9 \\ 6 \\ 0\end{array}\right]$, and $x=\left[\begin{array}{l}x_1 \\ x_2\end{array}\right]$ : \\~\\
Calculate: (a) $AI \quad$ (b) $I A \quad$ (c) $I x \quad$ (d) $x^{\prime} I$ \\
Indicate the dimension of the identity matrix used in each case.

% Question 4
\item[4.] Show that the diagonal matrix
$$
\left[\begin{array}{cccc}
a_{11} & 0 & \cdots & 0 \\
0 & a_{22} & \cdots & 0 \\
\cdots  & \cdots & \cdots & \cdots  \\
0 & 0 & \cdots & a_{n n}
\end{array}\right]
$$
can be idempotent only if each diagonal element is either 1 or 0 . How many different numerical idempotent diagonal matrices of dimension $n \times n$ can be constructed altogether from such a matrix? \\
\end{itemize}

%%%%% Exercise $4.6$
\underline{Exercise $4.6$} 
\begin{itemize}

% Question 2
\item[2.] Given $A=\left[\begin{array}{rr}0 & 4 \\ -1 & 3\end{array}\right], B=\left[\begin{array}{rr}3 & -8 \\ 0 & 1\end{array}\right]$, and $C=\left[\begin{array}{lll}1 & 0 & 9 \\ 6 & 1 & 1\end{array}\right]$, verify that
\begin{enumerate}
\item[(a)] $(A+B)^{\prime}=A^{\prime}+B^{\prime}$
\item[(b)] $(A C)^{\prime}=C^{\prime} A^{\prime}$ 
\end{enumerate}

% Question 6
\item[6.]  Let \( A=I-X\left(X^{\prime} X\right)^{-1} X^{\prime} \).
\begin{enumerate}
\item[(a)] Must $A$ be square? Must $\left(X^{\prime} X\right)$ be square? Must $X$ be square?
\item[(b)] Show that matrix $A$ is idempotent.
\end{enumerate}
\end{itemize}

\end{document}