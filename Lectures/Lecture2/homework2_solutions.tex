\documentclass{./../../Latex/handout}
\begin{document}
\thispagestyle{plain}
\newcommand{\mytitle}{Homework 2 Solutions}
\myheader{\mytitle}

%%%%% Exercise $4.2$
\underline{Exercise $4.2$}

\begin{enumerate}	
	% Question 1
  \item $A=\left[\begin{array}{cc}7 & -1 \\ 6 & 9\end{array}\right] \quad B=\left[\begin{array}{cc}0 & 4 \\ 3 & -2\end{array}\right] \quad C=\left[\begin{array}{ll}8 & 3 \\ 6 & 1\end{array}\right]$
\begin{enumerate}
\item  $A+B=\left[\begin{array}{ll}7 & 3 \\ 9 & 7\end{array}\right]$
\item  $C-A=\left[\begin{array}{cc}1 & 4 \\ 0 & -8\end{array}\right]$
\item  $3 A=\left[\begin{array}{cc}21 & -3 \\ 18 & 27\end{array}\right]$
\item  $4 B+2 C=\left[\begin{array}{cc}0 & 16 \\ 12 & -8\end{array}\right]+\left[\begin{array}{cc}16 & 6 \\ 12 & 2\end{array}\right]=\left[\begin{array}{cc}16 & 22 \\ 24 & -6\end{array}\right]$ \\~\\
\end{enumerate} 

% Question 2
\item $A = \left[\begin{array}{cc}
2 & 8 \\
3 & 0 \\
5 & 1
\end{array}\right] \quad B=\left[\begin{array}{cc} 2 & 0 \\ 3 & 8\end{array}\right] \quad
C=\left[\begin{array}{cc} 7 & 2 \\ 6 & 3\end{array}\right]$

\begin{enumerate}
\item $A B$ is defined as number of columns in $A$ is two which is equal to the number of rows in $B$.
$$
A B=\left[\begin{array}{cc}
2 \times 2+8 \times 3 & 2 \times 0+8 \times 8 \\
3 \times 2+0 \times 3 & 3 \times 0+0 \times 8 \\
5 \times 2+1 \times 3 & 5 \times 0+1 \times 8
\end{array}\right]=\left[\begin{array}{cc}
28 & 64 \\
6 & 0 \\
13 & 8
\end{array}\right]
$$
Not possible to calculate BA as $B$ has two columns, but $A$ has three rows.

\item $B C$ and $C B$ are both defined as both have two rows and two columns.
$$
B C=\left[\begin{array}{ll}
14 & 4 \\
69 & 30
\end{array}\right] \neq C B=\left[\begin{array}{cc}
20 & 16 \\
21 & 24
\end{array}\right]
$$ \\
\end{enumerate} 

% Question 4
\item[4.] \begin{tasks}(2)
\task[(a)] $\left[\begin{array}{cc}0 & 2 \\ 36 & 20 \\ 16 & 3\end{array}\right]_{3 \times 2}$
\task[(b)] $\left[\begin{array}{ll}49 & 3 \\ 4 & 3\end{array}\right]_{2 \times 2}$
\task[(c)] $\left[\begin{array}{l}3 x+5 y \\ 4 x+2 y-7 z\end{array}\right]_{2 \times 1}$ 
\task[(d)] $[7 a+c \quad 2 b+4 c]$ \\~\\
\end{tasks} 
\end{enumerate} 


%%%%% Exercise $4.4$
\underline{Exercise $4.4$}

% Question 5 (e)
\begin{enumerate}
\item[5.] (e)  $$ \begin{aligned}
 & A=\left[\begin{array}{c}
-2 \\
4 \\
7
\end{array}\right]_{3 \times 1} \quad B=\left[\begin{array}{lll}
3 & 6 & -2
\end{array}\right]_{1 \times 3} \\~\\
& C=A B=\left[\begin{array}{ccc}
-6 & -12 & 4 \\
12 & 24 & -8 \\
21 & 42 & -14
\end{array}\right] \\~\\
& D=B A=[3 \times-2+6 \times 4+-2 \times 7]=[4]
\end{aligned} $$\\~\\

% Question 7
\item[7.] In example 5, $
\begin{aligned} x=\left[\begin{array}{l}x_{1} \\x_{2}\end{array}\right] \text{ and }A=\left[\begin{array}{cc}a_{11} & 0 \\0 & a_{22}\end{array}\right] \end{aligned} $.
In which case, 
$$
\begin{aligned}
& x^{\prime} A x=\left[\begin{array}{ll}x_{1} & x_{2}\end{array}\right]\left[\begin{array}{cc}a_{11} & 0 \\0 & a_{22}\end{array}\right]\left[\begin{array}{l}x_{1} \\x_{2}\end{array}\right]_{2 \times 1} \\
& =\left[\begin{array}{lll}a_{11} x_{1} & a_{22} x_{2}\end{array}\right]_{1 \times 2}\left[\begin{array}{l}x_{1} \\x_{2}\end{array}\right]_{2 \times 1} \\
& =\underbrace{a_{11} x_{1}^{2}+a_{22} x_{2}^{2}}_{\text{Weighted sum of squares}}=\sum_{i=1}^{2} a_{i i} x_{i}^{2}
\end{aligned}
$$

So $x^{\prime} A x$ represents a weighted sum of squares where $a_{11}, a_{22}$ are weights.\\

But now what if $A=\left[\begin{array}{ll}a_{11} & a_{12} \\ a_{21} & a_{22}\end{array}\right]$. In this case, 

$$
x^{\prime} A x=\left[\begin{array}{ll}
x_{1} & x_{2}
\end{array}\right]\left[\begin{array}{ll}
a_{11} & a_{12} \\
a_{21} & a_{22}
\end{array}\right]_{2 \times 2}\left[\begin{array}{l}
x_{1} \\
x_{2}
\end{array}\right]_{2 \times 1}
$$

$$
\begin{aligned}
& =\left[\begin{array}{rr}
a_{11} x_{1}+a_{21} x_{2} & a_{12} x_{1}+a_{22} x_{2}
\end{array}\right]\left[\begin{array}{l}
x_{1} \\
x_{2}
\end{array}\right]_{2 \times 1} \\
& =a_{11} x_{1}^{2}+a_{21} x_{1} x_{2}+a_{12} x_{1} x_{2}+a_{22} x_{2}^{2} \\
& =a_{11} x_{1}^{2}+\left(a_{21}+a_{12}\right) x_{1} x_{2}+a_{22} x_{2}^{2}
\end{aligned}
$$

So $x^{\prime} A x$ no longer represents a weighted sum of squares. \\

You can check that the associative law i.e.
$$
\left(x^{\prime} A\right) x=x^{\prime}(A x)
$$
will apply in both cases (after all, its a law!) as all products are possible.\\~\\

\newpage
%%%%% Exercise $4.5$
\underline{Exercise $4.5$}
	
% Question 1
\item[1.] $$A=\left[\begin{array}{ccc}-1 & 5 & 7 \\0 & -2 & 4\end{array}\right]_{2 \times 3} \quad \quad B=\left[\begin{array}{c}9 \\6 \\0\end{array}\right]_{3 \times 1} \quad \quad x=\left[\begin{array}{l}x_{1} \\x_{2}\end{array}\right]_{2\times 1}$$ 

\begin{enumerate}

\item $ \begin{aligned} A I=\left[\begin{array}{ccc}-1 & 5 & 7 \\0 & -2 & 4\end{array}\right]\left[\begin{array}{lll}1 & 0 & 0 \\0 & 1 & 0 \\0 & 0 & 1\end{array}\right]_{3 \times 3}  =\left[\begin{array}{ccc}-1 & 5 & 7 \\0 & -2 & 4\end{array}\right]=A \end{aligned}$  \\~\\

\item $ \begin{aligned} IA =\left[\begin{array}{ll}
1 & 0 \\
0 & 1
\end{array}\right]_{2\times 2}\left[\begin{array}{ccc}
-1 & 5 & 7 \\
0 & -2 & 4
\end{array}\right]  =\left[\begin{array}{rrr}
-1 & 5 & 7 \\
0 & -2 & 4
\end{array}\right] \end{aligned}$ \\~\\

\item $ \begin{aligned} Ix=\left[\begin{array}{ll}1 & 0 \\ 0 & 1\end{array}\right]_{2 \times 2}\left[\begin{array}{l}x_{1} \\ x_{2}\end{array}\right]=\left[\begin{array}{l}x_{1} \\ x_{2}\end{array}\right]\end{aligned}$ \\~\\

\item $ \begin{aligned} x^{\prime} I=\left[\begin{array}{ll}x_{1} & x_{2}\end{array}\right]\left[\begin{array}{ll}1 & 0 \\ 0 & 1\end{array}\right]_{2 \times 2}=\left[\begin{array}{ll}x_{1} & x_{2}\end{array}\right]\end{aligned}$ \\~\\
\end{enumerate}

% Question 4
\item[4.]Let's start with a $2 \times 2$ diagonal matrix

$$
\left[\begin{array}{cc}
a_{11} & 0 \\
0 & a_{22}
\end{array}\right]\left[\begin{array}{cc}
a_{11} & 0 \\
0 & a_{22}
\end{array}\right]=\left[\begin{array}{cc}
a_{11}^{2} & 0 \\
0 & a_{22}^{2}
\end{array}\right]
$$

$x=x^{2}$ for only $x=0,1$ so $a_{11}$ and $a_{22}$ can either be 0 or 1 . So we can have the following $2 \times 2$ idempotent diagonal matrices:

$$
\left[\begin{array}{ll}
1 & 0 \\
0 & 0
\end{array}\right],\left[\begin{array}{ll}
1 & 0 \\
0 & 1
\end{array}\right],\left[\begin{array}{ll}
0 & 0 \\
0 & 0
\end{array}\right],\left[\begin{array}{ll}
0 & 0 \\
0 & 1
\end{array}\right]
$$

More generally, for $n \times n$ matrix, there can be $2^{n}$ such matrices. This is because there will be $n$ elements, each of which can take two values.
\end{enumerate}

%%%%%%%%%% Exercise 4.6
\underline{Exercise 4.6}  \\

2. $
A=\left[\begin{array}{cc}
0 & 4 \\
-1 & 3
\end{array}\right] \quad B=\left[\begin{array}{cc}
3 & -8 \\
0 & 1
\end{array}\right] \quad C=\left[\begin{array}{lll}
1 & 0 & 9 \\
6 & 1 & 1
\end{array}\right]
$ \\

\begin{itemize}

\item[(a)] \( A+B=\left[\begin{array}{cc}3 & -4 \\ -1 & 4\end{array}\right] \) \\~\\
\( A^{\prime}+B^{\prime}=\left[\begin{array}{rr}0 & -1 \\ 4 & 3\end{array}\right]+\left[\begin{array}{cc}3 & 0 \\ -8 & 1\end{array}\right]=\left[\begin{array}{cc}3 & -1 \\ -4 & 4\end{array}\right] \) \\~\\
So, \((A+B)^{\prime}=A^{\prime}+B^{\prime}\) \\


\item[(b)] \(
A C=\left[\begin{array}{cc}
0 & 4 \\
-1 & 3
\end{array}\right]_{2 \times 2}\left[\begin{array}{lll}
1 & 0 & 9 \\
6 & 1 & 1
\end{array}\right]_{2 \times 3}=\left[\begin{array}{ccc}
24 & 4 & 4 \\
17 & 3 & -6
\end{array}\right]_{2 \times 3}
\)
\\~\\
\(
C^{\prime} A^{\prime}=\left[\begin{array}{ll}
1 & 6 \\
0 & 1 \\
9 & 1
\end{array}\right]_{3 \times 2}\left[\begin{array}{cc}
0 & -1 \\
4 & 3
\end{array}\right]_{2 \times 2}=\left[\begin{array}{cc}
24 & 17 \\
4 & 3 \\
4 & -6
\end{array}\right]_{3 \times 2}
\) \\~\\
So, $(AC)^{\prime}=C^{\prime} A^{\prime}$.
\end{itemize}

6. \( A=I-X\left(X^{\prime} X\right)^{-1} X^{\prime} \) 

\begin{itemize}

\item[(a.)] Say the dimension of $X$ is \( m \times n \). Then the dimension of \( X_{n \times m}^{\prime} X_{m \times n} \) is \( n \times n \). So the dimension of \( \left(X^{\prime} X\right)^{-1} \) is also  \( n \times n \).  This implies that the dimension of \( X_{m \times n}\left(X^{\prime} X\right)_{n \times n}^{-1} X_{n \times m}^{\prime} \) is \( m \times m \). Hence, \( X^{\prime} X \) and $A$ must be square matrices, but $X$ need not be square.

\item[(b.)] To prove a matrix is idempotent, we need to show $AA=A$.
$$
\begin{aligned}
A A &= (I-X\left(X^{\prime} X\right)^{-1} X^{\prime})(I-X\left(X^{\prime} X\right)^{-1} X^{\prime}) \\
& = I-X\left(X^{\prime} X\right)^{-1} X^{\prime}-X\left(X^{\prime} X\right)^{-1} X^{\prime}+X\underbrace{\left(X^{\prime} X\right)^{-1} X^{\prime} X}_{I}\left(X^{\prime} X\right)^{-1} X^{\prime} \\
&= I-X\left(X^{\prime} X\right)^{-1} X^{\prime}-X\left(X^{\prime} X\right)^{-1} X^{\prime}+X\left(X^{\prime} X\right)^{-1} X^{\prime} \\ 
&=I-X\left(X^{\prime} X\right)^{-1} X^{\prime} = A
\end{aligned}
$$
\end{itemize}

\end{document}