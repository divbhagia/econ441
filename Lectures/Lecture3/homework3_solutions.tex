\documentclass{./../../Latex/homework}

\begin{document}

\thispagestyle{plain}
\newcommand{\mytitle}{Homework 3 Solutions}
\myheader{\mytitle}

%%%%%%%%%%%%%% Exercise 5.1
\subsection*{Exercise 5.1}  
\begin{enumerate}

% Question 3
\item[3.] \begin{tasks}(2)
\task Yes
\task Yes
\task Yes
\task No, the second row $=$ 2 $\times$ first row. \\
\end{tasks}

% Question 4
\item[4.] Yes, we get the same answer. \\

% Question 5
\item[5.]
\begin{enumerate}
\item[(a)]  \[
A=\left[\begin{array}{ccc}
1 & 5 & 1 \\
0 & 3 & 9 \\
-1 & 0 & 0
\end{array}\right]
\]
Exchange Row 2 and 3
\[
A_{1}=\left[\begin{array}{ccc}
1 & 5 & 1 \\
-1 & 0 & 0 \\
0 & 3 & 9
\end{array}\right]
\]

New Row 2 = Row 1 $+$ Row 2
\[
A_{2}=\left[\begin{array}{lll}
1 & 5 & 1 \\
0 & 5 & 1 \\
0 & 3 & 9
\end{array}\right]
\]

New Row 3 = Row 3$-$3/5 $\times$ Row 2
\[
A_{3}=\left[\begin{array}{lll}
1 & 5 & 1 \\
0 & 5 & 1 \\
0 & 0 & \frac{42}{5}
\end{array}\right]
\]
The resulting echelon matrix $A_3$ contains 3 nonzero rows and hence $A$ has rank 3. Since $A$ is full-rank, it is a nonsingular matrix.  \\~\\

\item[(b)]  \[
B=\left[\begin{array}{rrr}
0 & -1 & -4 \\
3 & 1 & 2 \\
6 & 1 & 0
\end{array}\right] 
\]
New Row 1 = Row 2, New Row 2 = Row 3, New Row 3 = Row 1
\[ 
B_{1}=\left[\begin{array}{rrr}
3 & 1 & 2 \\
6 & 1 & 0 \\
0 & -1 & -4
\end{array}\right] 
\]
New Row 2 = Row 2$-$2 $\times$ Row 1
$$
B_{2}=\left[\begin{array}{rrr}
3 & 1 & 2 \\
0 & -1 & -4 \\
0 & -1 & -4
\end{array}\right]
$$
New Row 3 = Row 3$-$Row 2

$$
B_{3}=\left[\begin{array}{rrr}
3 & 1 & 2 \\
0 & -1 & -4 \\
0 & 0 & 0
\end{array}\right]
$$
The resulting echelon matrix $B_3$ contains only 2 nonzero rows and hence $B$ has rank 2. Since the rank of $B$ is less than the number of rows and columns, $B$ is a singular matrix.  \\~\\

\item[(c)] $$
C =\left[\begin{array}{llll}
7 & 6 & 3 & 3 \\
0 & 1 & 2 & 1 \\
8 & 0 & 0 & 8
\end{array}\right]
$$ 
Interchange rows 2 and 3.
$$
C_1 =\left[\begin{array}{rrrr}
7 & 6 & 3 & 3 \\
8 & 0 & 0 & 8 \\
0 & 1 & 2 & 1 \\
\end{array}\right]
$$ 
New Row 2 = Row 2$-\frac{8}{7} \times$Row 1
\[
C_2 =\left[\begin{array}{rrrr}
7 & 6 & 3 & 3 \\
0 & -\frac{48}{7} &  -\frac{24}{7} & 8-\frac{24}{7}  \\
0 & 1 & 2 & 1 \\
\end{array}\right]
\] 
\[
C_2 =\left[\begin{array}{rrrr}
7 & 6 & 3 & 3 \\
0 & -\frac{48}{7} &  -\frac{24}{7} & \frac{32}{7}  \\
0 & 1 & 2 & 1 \\
\end{array}\right]
\]
New Row 3 = Row 3$+\frac{7}{48}$ Row 2
$$
C_3 =\left[\begin{array}{rrrr}
7 & 6 & 3 & 3 \\
0 & -\frac{48}{7} &  -\frac{24}{7} & \frac{32}{7}  \\
0 & 0 & \frac{3}{2} & \frac{5}{3} \\
\end{array}\right]
$$ 
Rank of $C$ is 3. The concept of nonsingularity is only defined for square matrices. \\~\\
 
\item[(d)] $$ D =\left[\begin{array}{rrrr}
2 & 7 & 9 & -1 \\
1 & 1 & 0 & 1 \\
0 & 5 & 9 & -3
\end{array}\right] $$
New Row 2$=$ Row 2$- 0.5 \times$ Row 1
$$ D_1 =\left[\begin{array}{rrrr}
2 & 7 & 9 & -1 \\
0 & -2.5 & -4.5 & 1.5 \\
0 & 5 & 9 & -3
\end{array}\right] $$
New Row 3$=$ Row 3 $+ 2  \times$ Row 2
$$ D_2 =\left[\begin{array}{rrrr}
2 & 7 & 9 & -1 \\
0 & -2.5 & -4.5 & 1.5 \\
0 & 0& 0 & 0
\end{array}\right] $$
Rank of $D$ is 2.
\end{enumerate}

% Question 6
\item[6.] In the end, converting the matrix to echelon form is trying to find if any combination of rows will lead to a sum of 0, which is the definition of linear independence. 
\end{enumerate}


%%%%%%%%%%%%%%%% Exercise 5.2
\subsection*{Exercise 5.2}

% Question 1
\begin{enumerate}
\item 
\begin{enumerate}
\item[(c)] $$
\begin{aligned}
& 8\left|\begin{array}{ll}
0 & 1 \\
0 & 3
\end{array}\right|-1\left|\begin{array}{ll}
4 & 1 \\
6 & 3
\end{array}\right|+3\left|\begin{array}{ll}
4 & 0 \\
6 & 0
\end{array}\right| \\
=& 8(0-0)-1(12-6)+3(0-0) \\
=& 0-6+0 \\
= &-6 \\
\end{aligned}
$$

\item[(e)] $$
\begin{aligned}
& a\left|\begin{array}{ll}
c & a \\
a & b
\end{array}\right|-b\left|\begin{array}{cc}
b & a \\
c & b
\end{array}\right|+c\left|\begin{array}{cc}
b & c \\
c & a
\end{array}\right| \\
=& a\left(c b-a^{2}\right)-b\left(b^{2}-a c\right)+c\left(a b-c^{2}\right) \\
=& a b c-a^{3}-b^{3}+a b c+a b c-c^{3} \\
=&-a^{3}-b^{3}-c^{3}+3 a b c
\end{aligned}
$$

\item[(f)] $$
\begin{aligned}
& x\left|\begin{array}{cc}
y & 2 \\
-1 & 8
\end{array}\right|-5\left|\begin{array}{cc}
3 & 2 \\
9 & 8
\end{array}\right|+0\left|\begin{array}{rr}
3 & y \\
9 & -1
\end{array}\right| \\
=& x(8 y+2)-5(24-18)+0(-3-9 y) \\
=& 8 x y+2 x-30+0 \\
=& 8 x y+2 x-30
\end{aligned}
$$
\end{enumerate}

% Question 2
  \item \( \left|C_{13}\right|: 1+3=4 \) is even so \( + \)\\
\( \left|C_{23}\right|: 2+3=5 \) is odd so \(- \)\\
\( \left|C_{33}\right|: 3+3=6 \) is even so \( + \)\\
\( \left|C_{41}\right|: 4+1=5 \) is odd so \( - \)\\
\( \left|C_{34}\right|: 3+4=7 \) is odd so \( - \) \\


% Question 3
\item 
Minor of \(a\):
$$ \left|M_{11}\right|=\left|\begin{array}{cc}e & f \\ h & i\end{array}\right|=e i-f h $$
Cofactor of \(a\): $$
\left|C_{11}\right|=(-1)^{1+1}\left|M_{11}\right|=\left|M_{11}\right|
$$

Minor of \(b\):
$$ \left|M_{12}\right|=\left|\begin{array}{ll}d & f \\ g & i\end{array}\right|=d i-f g $$
Cofactor of \(b\):
$$\left|C_{12}\right|=(-1)^{3}\left|M_{12}\right|=-\left|M_{12}\right| $$
Minor of \(f\):
$$ \left|M_{23}\right|=\left|\begin{array}{ll}a & b \\ g & h\end{array}\right|=a h-b g $$
Cofactor of \(f\):
$$\left|C_{23}\right|=(-1)^{5}\left|M_{23}\right|=-\left|M_{23}\right| $$ \\

% Question 6
\item[6.] Minors of third row:
\[
\begin{aligned}
\left|M_{31}\right| &=\left|\begin{array}{cc}
11 & 4 \\
2 & 7
\end{array}\right| = 69, \quad
\left|M_{32}\right|&=\left|\begin{array}{cc}
9 & 4 \\
3 & 7
\end{array}\right| = 51, \quad
 \left|M_{33}\right|&=\left|\begin{array}{ll}
9 & 11 \\
3 & 2
\end{array}\right| =-15
\end{aligned}
\]
Cofactors: $ |C_{31}| =  |M_{31}|, |C_{32}| =  -|M_{32}| , |C_{33}| =  |M_{33}| $.\end{enumerate}

\end{document}