\documentclass{./../../Latex/handout}
\begin{document}
\thispagestyle{plain}
\newcommand{\mytitle}{Solving a System of Equations}
\myheader{\mytitle}

Given the system of equations:
$$
\begin{array}{l}
4 x_{1}+3 x_{2}=28 \\
2 x_{1}+5 x_{2}=42
\end{array}
$$


Solve the above system of equations using matrix algebra. Start by writing out the equations in matrix format: 
$$ Ax = b $$
where $A$ is the coefficient matrix, $x$ is the vector of unknowns, and $b$ is the vector of constants. 

Then solve the equations using two methods:
\begin{enumerate}
\item Using the inverse of a matrix i.e. $$x^*= A^{-1} b$$
\item Using Cramer's rule $$x^*_k = \frac{|A_k|}{|A|}$$
Here, $A_k$ is the matrix formed by interchanging the $k^{th}$ column of $A$ by $b$.
\end{enumerate} 

 

\end{document}