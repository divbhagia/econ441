\documentclass{./../../Latex/homework}
\begin{document}
\thispagestyle{plain}
\newcommand{\mytitle}{Homework 4 Problems}
\myheader{\mytitle}

%%%%%%%%%%%%%% Exercise 5.3
\subsection*{Exercise 5.3}

\begin{enumerate}

% Question 1
\item[1.] Use the determinant $\left|\begin{array}{rrr}4 & 0 & -1 \\ 2 & 1 & -7 \\ 3 & 3 & 9\end{array}\right|$ to verify the first four properties of determinants.

% Question 4
\item[4.] Show that when all the elements of an $n$th-order determinant |A| are multiplied by a number $k$, the result will be $k^{n}|A|$.

% Question 5
\item[5.] Calculate the determinant for the following matrices. Comment on whether the matrices are nonsingular and the rank of each matrix.
\begin{tasks}(2)
\task $\left[\begin{array}{rrr}4 & 0 & 1 \\ 19 & 1 & -3 \\ 7 & 1 & 0\end{array}\right]$
\task $\left[\begin{array}{rrr}4 & -2 & 1 \\ -5 & 6 & 0 \\ 7 & 0 & 3\end{array}\right]$
\task $\left[\begin{array}{rrr}7 & -1 & 0 \\ 1 & 1 & 4 \\ 13 & -3 & -4\end{array}\right]$
\task $\left[\begin{array}{rrr}-4 & 9 & 5 \\ 3 & 0 & 1 \\ 10 & 8 & 6\end{array}\right]$	
\end{tasks}

% Question 8
\item[8.] Comment on the validity of the following statements:
\begin{enumerate}
\item Given any matrix A, we can always derive from it a transpose and a determinant.
\item Multiplying each element of an $n \times n$ determinant by 2 will double the value of that determinant.
\item If a square matrix $A$ vanishes, then we can be sure that the equation system $A x=d$ is nonsingular.
\end{enumerate}
\end{enumerate}

%%%%%%%%%%%%%% Exercise 5.4
\subsection*{Exercise 5.4}

\begin{enumerate}

% Question 2
\item[2.] Find the inverse of each of the following matrices:
\begin{tasks}(2)
\task $A=\left[\begin{array}{ll}5 & 2 \\ 0 & 1\end{array}\right]$
\task $B=\left[\begin{array}{rr}-1 & 0 \\ 9 & 2\end{array}\right]$
\task $C=\left[\begin{array}{rr}3 & 7 \\ 3 & -1\end{array}\right]$
\task $D=\left[\begin{array}{ll}7 & 6 \\ 0 & 3\end{array}\right]$ 
\end{tasks} 

% Question 3
\item[3.] 
\begin{enumerate}
\item Drawing on your answers to Prob. 2, formulate a two-step rule for finding the adjoint of a given $2 \times 2$ matrix $A$: In the first step, indicate what should be done to the two diagonal elements of $A$ in order to get the diagonal elements of $adjA$; in the second step, indicate what should be done to the two off-diagonal elements of $A$. (Warning: This rule applies only to $2 \times 2$ matrices.) 
\item Add a third step which, in conjunction with the previous two steps, yields the $2 \times 2$ inverse matrix $A^{-1}$.
\end{enumerate}

% Question 4
\item[4.] Find the inverse of each of the following matrices:
\begin{tasks}(2)
\task $E=\left[\begin{array}{rrr}4 & -2 & 1 \\ 7 & 3 & 0 \\ 2 & 0 & 1\end{array}\right]$
\task $F=\left[\begin{array}{rrr}1 & -1 & 2 \\ 1 & 0 & 3 \\ 4 & 0 & 2\end{array}\right]$
\task $G=\left[\begin{array}{lll}1 & 0 & 0 \\ 0 & 0 & 1 \\ 0 & 1 & 0\end{array}\right]$
\task $H=\left[\begin{array}{lll}1 & 0 & 0 \\ 0 & 1 & 0 \\ 0 & 0 & 1\end{array}\right]$
\end{tasks} 

% Question 6
\item[6.] Solve the system $A x=d$ by matrix inversion, where
\vspace{-1.25cm}
\begin{tasks}(2)
\task  $\begin{aligned} \\
	4 x+3 y=28 \\
	2 x+5 y=42
\end{aligned}$
\task $\begin{aligned} \\~\\
	4 x_{1}+x_{2}-5 x_{3}=8 \\
	-2 x_{1}+3 x_{2}+x_{3}=12 \\
	3 x_{1}-x_{2}+4 x_{3}=5
\end{aligned}$
\end{tasks} 


% Question 7
\item[7.] Is it possible for a matrix to be its own inverse?

\end{enumerate}


%%%%%%%%%%%%%% Exercise 5.5
\subsection*{Exercise 5.5}

\begin{enumerate}

% Question 1
\item[1.] Use Cramer's rule to solve the following equation systems:
\vspace{-0.75cm}
\begin{tasks}(2)
\task  $\begin{aligned} \\
	3x_1-2x_2 = 6 \\
	2x_1 + x_2 =11
\end{aligned}$  \vspace{-0.75cm}
\task $\begin{aligned} \\
	-x_1+3x_2 =-3 \\
	4x_1-x_2 = 12 
\end{aligned}$ \vspace{-0.75cm}
\task $\begin{aligned} \\
	8x_1-7x_2 =9 \\
	x_1 + x_2 = 3
\end{aligned}$ \vspace{-0.75cm}
\task $\begin{aligned} \\
	5x_1 + 9x_2 = 14 \\
	7 x_1-3x_2 = 4
\end{aligned}$ 
\end{tasks} 

% Question 2
\item[2.] For each of the equation systems in Prob. 1, find the inverse of the coefficient matrix, and get the solution by the formula $x^{*}=A^{-1} d$.

% Question 3
\item[3.] Use Cramer's rule to solve the following equation systems:
\vspace{-1.25cm}
\begin{tasks}(2)
\task[(a)]  $\begin{aligned} \\~\\
	8 x_{1}-x_{2}=16 \\
	2x_2 + 5 x_3 = 5 \\
	2x_1 + 3x_3 = 7
\end{aligned}$  \vspace{-1.25cm}
\task[(d)] $\begin{aligned} \\~\\
	-x + y+z = a \\
	x-y + z = b \\
	x+y-z = c
\end{aligned}$ 
\end{tasks} 

\end{enumerate}


\end{document}