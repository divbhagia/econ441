\documentclass{./../../Latex/handout}
\begin{document}
\thispagestyle{plain}
\newcommand{\mytitle}{Rules of Differentiation}
\myheader{\mytitle}

%$$
%\begin{aligned}
%\frac{d y}{d x} &=\frac{2(x+1)-(2 x-3) 1}{(x+1)^{2}} \\
%&=\frac{2 x+2-2 x+3}{(x+1)^{2}} \\
%&=\frac{5}{(x+1)^{2}}
%\end{aligned}
%$$
\vspace{-2em}
\section*{Function of a Single Variable} 
\vspace{-1em}
\begin{itemize}
  \item \underline{Constant function rule:} If \(f(x)=k \), \( f'(x)=0 \). 
  \item \underline{Power function rule:} If \(f(x)=x^n \), \( f'(x)=n x^{n-1} \). 
  \item \underline{Generalized power function rule:} If \(f(x)=c x^{n} \), \( f'(x)=cn x^{n-1} \). 
  
  \textit{Example}. For $y=f(x) = 3x^2$, $f'(x)= 6x$.
  \item \underline{Inverse function rule:} Given $y = f(x)$ and inverse function $x = f^{-1}(y)$
$$
\frac{d x}{d y}=\frac{1}{d y/ d x}
$$ 
\end{itemize}
\textit{Exercises}
\begin{enumerate}
	\item Find the derivative of $y=-3x^6$? \\~\\ \vspace{1.25cm}
	\item Verify the inverse function rule for $y=6x+1$. \\~\\ \vspace{2.5cm}
	\item Find the derivative of $y=1/x$. 
\end{enumerate}
\newpage
In the class, we learned the limit definition of the derivative. In particular,
$$ \frac{d y}{d x}=\lim _{\Delta x \rightarrow 0} \frac{\Delta y}{\Delta x} =\lim _{\Delta x \rightarrow 0} \frac{f(x+\Delta x)-f(x)}{\Delta x} $$

All the rules of differentiation can actually be derived from the limit definition of the derivative. But it is easier to have a set of rules to quickly differentiate a function rather than writing down the limit each time. \\

For {example}, if we have the function $y=ax$, using the limit definition of the derivative: 
$$ \begin{aligned} \frac{d y}{d x} &=\lim _{\Delta x \rightarrow 0} \frac{a(x+\Delta x)-a x}{\Delta x} \\ &=\lim _{\Delta x \rightarrow 0} a=a \end{aligned} $$

Similarly, for $y=ax^2$
$$
\begin{aligned}
\frac{\Delta y}{\Delta x} &=\frac{a(x+\Delta x)^{2}-a x^{2}}{\Delta x} \\
&=\frac{a x^{2}+a \Delta x^{2}+2 a x \Delta x-a x^{2}}{\Delta x} \\
&=a \Delta x+2a x \\~\\
\frac{d y}{d x}&=\lim _{\Delta x \rightarrow 0} \frac{\Delta y}{\Delta x} =2 ax
\end{aligned}
$$

\textit{Exercise. Verify the constant function rule using the limit definition.}

\newpage
\section*{Two or More Functions of a Single Variable}
\begin{itemize}
\item \underline{Sum-Difference Rule}
$$ \frac{d}{d x}[f(x) \pm g(x)]=f^{\prime}(x) \pm g^{\prime}(x) $$
\textit{Example}. For $y=2x+x^2- x^{3}$, the derivative is given by $ \dfrac{d y}{d x}=2+2x-3x^2 $ \\

\item \underline{Product Rule}
$$
\frac{d}{d x}[f(x) g(x)]=f(x) g^{\prime}(x)+f^{\prime}(x) g(x)
$$ 

\textit{Example}. For  \( y =3 x\left(x^{2}+1\right) \)\\
\[
\begin{aligned}
\frac{dy}{d x} &=3 x(2 x)+3\left(x^{2}+1\right) \\
&=6 x^{2}+3x^2+3 = 9 x^2+3
\end{aligned}
\]\\
Alternatively, one can write \( y =3x^3+3x \) and evaluate the derivative directly. \\
\item \underline{Quotient Rule}
$$ \frac{d}{d x} \frac{f(x)}{g(x)}=\frac{f^{\prime}(x) g(x)-f(x) g^{\prime}(x)}{g(x)^2} $$ \\
\textit{Example}. Given the function $$
y=\frac{2 x-3}{x+1}
$$

We can calculate the derivative using the quotient rule as follows:
$$
\begin{aligned}
\frac{d y}{d x} &=\frac{2(x+1)-(2 x-3) 1}{(x+1)^{2}} \\
&=\frac{2 x+2-2 x+3}{(x+1)^{2}}=\frac{5}{(x+1)^{2}}
\end{aligned}
$$
\end{itemize}

\newpage
\textit{Exercise. Calculate the derivative for the following functions using product or quotient rule:}
$$ y = x(x+1), \quad y=\frac{1}{x}, \quad y = \frac{2x^3-x}{x^2}$$

\newpage 
\section*{Functions of Different Variables}

\underline{Chain Rule} 

If we have two functions:
$$ z=f(y), \quad y=g(x) $$
Then, 
$$\frac{d z}{d x}=\frac{d z}{d y} \cdot \frac{d y}{d x}=f^{\prime}(y) g^{\prime}(x) $$

\textit{Example}. $R = p Q $ is revenue from the sale of quantity $Q$ at price $p$. Quantity produced $Q= a L$ depends on labor input $L$. Then,
$$
\begin{aligned}
\frac{d R}{d L} &=\frac{d R}{d Q} \cdot \frac{d Q}{d L} = p \cdot a\\
\end{aligned}
$$


\textit{Exercise.} Find the derivative of $f$ with respect to $x$ where:
$$ f(y) = y^2-1 \quad \quad y = 2x^2  $$

\vspace{3cm}
\textit{Exercise.} Find the derivative of $y = \dfrac{1}{(2x+1)^3}$ using Chain Rule by defining the outer function $y = f(u) = \dfrac{1}{u^3}$ and inner function $u = g(x) = 2x+1$. 

\newpage

If you are bored:

\textit{Exercise. Verify the product rule using the limit definition of the derivative.} 

Easier if you start with the following definition:
$$ h'(x)=\lim _{x_0 \rightarrow x}  \frac{h(x)-h(x_0)}{x-x_0} $$\\
(Hint: Start by adding and subtracting $f(x)g(x_0)$ in the numerator.)
\end{document}