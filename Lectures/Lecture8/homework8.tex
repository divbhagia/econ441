\documentclass{./../../Latex/homework}
\begin{document}
\thispagestyle{plain}
\newcommand{\mytitle}{Homework 8 Problems}
\myheader{\mytitle}

%%%%%%%%%%%%%% Exercise 9.2
\subsection*{Exercise 9.2}

\begin{enumerate}

% Question 1
\item[1.] Find the stationary values of the following (check whether they are relative maxima or minima or inflection points), assuming the domain to be the set of all real numbers:
\begin{itemize}
\item[(c)] $y=3 x^{2}+3$
\end{itemize}

% Question 2
\item[2.] Find the stationary values of the following (check whether they are relative maxima or minima or inflection points $)$, assuming the domain to be the interval $[0, \infty)$ 
\begin{itemize}
\item[(a)] $y=x^{3}-3 x+5$
\end{itemize}

% Question 3
\item[3.] Show that the function $y=x+1 / x$ (with $x \neq 0$ ) has two relative extrema, one a maximum and the other a minimum. is the "minimum" larger or smaller than the "maximum"? How is this paradoxical result possible?

% Question 4
\item[4.] Let $T=\phi(x)$ be a total function (e.g., total product or total cost):
\begin{enumerate}
  \item Write out the expressions for the marginal function $M$ and the average function $A$.
  \item Show that, when $A$ reaches a relative extremum, $M$ and $A$ must have the same value.
  \item What general principle does this suggest for the drawing of a marginal curve and an average curve in the same diagram?
  \item What can you conclude about the elasticity of the total function $T$ at the point where $A$ reaches an extreme value?
\end{enumerate}


\end{enumerate}

%%%%%%%%%%%%%% Exercise 9.3
\subsection*{Exercise 9.3}

\begin{enumerate}

% Question 2
\item[2.] Which of the following quadratic functions are strictly convex?
\begin{tasks}(2)
\task $y=9 x^{2}-4 x+8$
\task  $w=-3 x^{2}+39$
\task  $u=9-2 x^{2}$
\task  $v=8-5 x+x^{2}$
\end{tasks}

% Question 3
\item[3.] Draw $(a)$ a concave curve which is not strictly concave, and $(b)$ a curve which qualifies simultaneously as a concave curve and a convex curve.

% Question 4
\item[4.] Given the function $y=a-\frac{b}{c+x} \quad(a, b, c>0: x \geq 0)$, determine the general shape of its graph by examining $(a)$ its first and second derivatives, $(b)$ its vertical intercept, and (c) the limit of $y$ as $x$ tends to infinity. If this function is to be used as a consumption function, how should the parameters be restricted in order to make it economically sensible?

% Question 5
\item[5.] Draw the graph of a function $f(x)$ such that $f^{\prime}(x) = 0$, and the graph of a function $g(x)$ such that $g^{\prime}(3)=0$. Summarize in one sentence the essential difference between $f(x)$ and $g(x)$ in terms of the concept of stationary point.

\end{enumerate}

%%%%%%%%%%%%%% Exercise 9.4
\subsection*{Exercise 9.4}

\begin{enumerate}

% Question 1
\item[1.] Find the relative maxima and minima of $y$ by the second-derivative test:
\begin{tasks}
\task[(b)] $y=x^{3}+6 x^{2}+9$
\task[(d)] $y=\frac{2 x}{1-2 x} \quad\left(x \neq \frac{1}{2}\right)$
\end{tasks}

% Question 2
\item[2.] Mr. Greenthumb wishes to mark out a rectangular flower bed, using a wall of his house as one side of the rectangle. The other three sides are to be marked by wire netting, of which he has only 64 ft available. What are the length $L$ and width $W$ of the rectangle that would give him the largest possible planting area? How do you make sure that your answer gives the largest, not the smallest area?

% Question 3
\item[3.] A firm has the following total-cost and demand functions:
$$C=\frac{1}{3} Q^{3}-7 Q^{2}+111 Q+50, \quad \quad Q=100-P$$
\begin{enumerate}
\item Does the total-cost function satisfy the coefficient restrictions of (9.5)? ($a, c, d>0, b<0, b^{2}<3 a c$)
\item Write out the total-revenue function $R$ in terms of $Q$.
\item Formulate the total-profit function $\pi$ in terms of $Q$.
\item Find the profit-maximizing level of output $Q^{*}$.
\item What is the maximum profit?
\end{enumerate}

% Question 5
\item[5.] A quadratic profit function $\pi(Q)=h Q^{2}+j Q+k$ is to be used to reflect the following assumptions:
\begin{enumerate}
\item If nothing is produced, the profit will be negative (because of fixed costs).
\item The profit function is strictly concave.
\item The maximum profit occurs at a positive output level $Q^{\star}$.
\end{enumerate}
What parameter restrictions are called for?

\end{enumerate}

\end{document}